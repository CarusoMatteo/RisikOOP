\documentclass[a4paper,12pt]{report}

\usepackage{alltt, fancyvrb, url}
\usepackage{graphicx}
\usepackage[utf8]{inputenc}
\usepackage{float}
\usepackage{xcolor}
\usepackage{hyperref}

% Questo commentalo se vuoi scrivere in inglese.
\usepackage[italian]{babel}

\usepackage[italian]{cleveref}

\title{
RisiKOOP \\
\begin{large}
Il gioco strategico per la conquista del mondo
\end{large}
}

\author{Matteo Caruso, Matteo Ceccarelli, Franceso Sacripante}
\date{\today}


\begin{document}

\maketitle

\tableofcontents

\chapter{Analisi}

\section{Descrizione e requisiti}

Il software mira a replicare il gioco Risiko, un gioco da tavolo di strategia a turni dove ogni giocatore controlla una squadra di unità allo scopo di completare un obiettivo determinato da una Carta Obiettivo pescata a inizio partita.
Questa richiederà di conquistare dei continenti, annientare un'altra armata oppure conquistare un certo numero di territori.
Il gioco inizia spartendo tutti i territori tra i giocatori e dà dei territori iniziali con cui rinforzarli.
Ogni turno turno, il giocatore otterrà vari carri armati da posizionare sui suoi territori.
Potrà poi attaccare territori adiacenti ai propri.
Se riesce a conquistare almeno uno stato otterrà una Carta Territorio, utilizzabile per giocare combo
\footnote{Le combo sono tris di carte. Saranno approfondite in dettaglio sucecssivamente.} al fine di ottenere ulteriori unità nei successivi turni.
Infine avrà l'opportunità di spostare delle unità fra i suoi territori.

\subsubsection{Requisiti funzionali}
\begin{itemize}
	\item Il software dovrà permettere di giocare a una semplice versione di Risiko.
	\item Ogni giocatore ha una sua Carta Obiettivo e varie Carte Territorio.
	\item L'attacco avviene tramite il tiro di dadi, il cui confronto ne determinerà l'esito.
\end{itemize}

\subsubsection{Requisiti non funzionali}
\begin{itemize}
	\item La mappa è selezionabile, scelta dai giocatori a inizio partita.
	\item I giocatori dovranno poter nascondere le proprie Carte Obiettivo e Territorio agli altri giocatori.
\end{itemize}

\section{Modello del Dominio}

Il gioco inizia con la selezione dei giocatori, del loro colore e della mappa.
Successivamente vengono assegnati i territori, ed è chiesto ai giocatori di posizionare le loro unità rimanenti in quei territori.
Ora inizia il game-loop del gioco, che si ripete fino a quando un giocatore non vince:
% TODO: aggiungi fasi
\begin{itemize}
	\item Fase di rinforzo.
	\item Fase di attacco.
	\item Fase di spostamento finale.
\end{itemize}

% TODO: UML obbligatorio di sole interfacce: Mappa, Giocatore, Carte, Fasi di gioco.

\chapter{Design}

\section{Architettura}

L'architettura del software è basata su un pattern Model-View-Controller (MVC).
L'entry point dell'applicazione è il \texttt{Controller}, che si occupa di avviare il model, che eredita da \texttt{GameManager}, e le view registrate, che eredtano da \texttt{RisikoView}.
Risiko è un software molto legato alla visualizzazione del gioco, quindi per favorire il \begin{itshape}Separation of Concerns\end{itshape}, il controller è diviso in sotto-controller: \texttt{DataAddingContrller} permette di impostare giocatori e la mappa; \texttt{DataRetrieveController} favorisce l'ottenimento di informazioni quali il giocatore corrente; \texttt{TurnManager} permette di gestire le fasi del gioco.

% TODO: UML delle interfacce per tutti i controller, il game manager e la risiko view.

\section{Design dettagliato}
% TODO: Design individuale da fare.

\chapter{Sviluppo}

\section{Testing automatizzato}
Il testing automatizzato è stato realizzato tramite JUnit, focalizzato principalmente sul model, come l'insermineto della mappa, la gestione dei giocatori, la validazione delle combo di carte, e sulla gestione delle fasi di gioco.

\section{Note di sviluppo}
% TODO: Note di sviluppo individuale da fare con permalink a github su feature di linguaggio di cui vai fiero.

\chapter{Commenti finali}

\section{Autovalutazione e lavori futuri}
% Obbligatorio individuale.

\section{Difficoltà incontrate e commenit per i docenti}
% Opzionale individuale.

\appendix
\chapter{Guida utente}
\section{Avviare la partita}
\section{Rinfori iniziali}
\subsection{Giocare le combo}
\section{Attaccare}
\section{Spostamento finale}

\appendix
\chapter{Esercitazioni di laboratorio}

\section{matteo.caruso7@studio.unibo.it}

\begin{itemize}
	\item Laboratorio 08: \url{https://virtuale.unibo.it/mod/forum/discuss.php?d=178723#p247198}
	\item Laboratorio 09: \url{https://virtuale.unibo.it/mod/forum/discuss.php?d=179154#p247764}
	\item Laboratorio 10: \url{https://virtuale.unibo.it/mod/forum/discuss.php?d=180101#p248784}
	\item Laboratorio 11: \url{https://virtuale.unibo.it/mod/forum/discuss.php?d=181206#p250854}
\end{itemize}

\section{matteo.ceccarelli@studio.unibo.it}
% TODO: Compila i laboratori che hai fatto e metti l'email giusta.
\begin{itemize}
	\item Laboratorio XX: \url{https://virtuale.unibo.it}
\end{itemize}

\section{franceso.sacripante@studio.unibo.it}
% TODO: Compila i laboratori che hai fatto e metti l'email giusta.
\begin{itemize}
	\item Laboratorio XX: \url{https://virtuale.unibo.it}
\end{itemize}


\bibliographystyle{alpha}
\bibliography{blbliography}
\end{document}