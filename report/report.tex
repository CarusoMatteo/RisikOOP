\documentclass[a4paper,12pt]{report}

\usepackage{alltt, fancyvrb, url}
\usepackage{graphicx}
\usepackage[utf8]{inputenc}
\usepackage{float}
\usepackage{xcolor}
\usepackage{hyperref}

% Questo commentalo se vuoi scrivere in inglese.
\usepackage[italian]{babel}

\usepackage[italian]{cleveref}

\title{
RisiK-OOP \\
\begin{large}
Il gioco strategico per la conquista del mondo
\end{large}
}

\author{Matteo Caruso, Matteo Ceccarelli, Franceso Sacripante}
\date{\today}


\begin{document}

\maketitle

\tableofcontents

\chapter{Analisi}

\section{Descrizione e requisiti}

Il software mira a replicare il gioco Risiko, un gioco da tavolo di strategia a turni dove ogni giocatore controlla una squadra di truppe allo scopo di completare un obiettivo determinato da una Carta Obiettivo pescata a inizio partita.
Questa richiederà di conquistare dei continenti, annientare un'altra armata oppure conquistare un certo numero di stati.
Il gioco inizia spartendo tutti gli stati tra i giocatori e dei carri armati iniziali.
Ogni turno turno, il giocatore otterrà vari carri armati da posizionare sui suoi territori.
Potrà poi attaccare stati confinanti ai propri territori oppure raggiungibili via percorsi navali.
Se riesce a conquistare almeno uno stato otterrà una Carta Territorio, utilizzabile per ottenere ulteriori carri armati nei successivi turni.
Infine avrà l'opportunità di spostare delle unità fra i suoi territori.

\subsubsection{Requisiti funzionali}
\begin{itemize}
	\item Il software dovrà permettere di giocare a una semplice versione di Risiko.
	\item La mappa è selezionabile, scelta dai giocatori a inizio partita.
	\item Ogni giocatore ha una sua Carta Obiettivo e varie Carte Territorio segrete.
	\item L'attacco avviene tramite il tiro di dadi, il cui confronto ne determinerà l'esito.
\end{itemize}

\subsubsection{Requisiti non funzionali}
\begin{itemize}
	\item Giocare contro una rudimentale intelligenza artificiale, in grado di simulare i comportamenti di un giocatore.
	\item Salvare la partita, in modo da poterla riprendere in futuro.
\end{itemize}


\end{document}